\chapter{Uvod}
\lhead{\emph{Uvod}}
\label{Chapter1} % For referencing the chapter elsewhere, use \ref{Chapter1} 

Ovaj seminar se bavi statistickim testovima, tj. provjeravanjem ili pobijanjem hipoteza. Pocetna premisa je jednostavna, imamo neku hipotezu koju zelimo provjeriti. Ona moze biti svakojaka:

\begin{itemize}
	\item na FERu je 99\% muskaraca 
	\item dnevno 1000 studenata FFZG idu u Cassandru
	\item Varianca bodova na SISu je 15
\end{itemize}

Kao sto vidimo u primjerima one testiraju vjerojatnost za neki paramater u populaciji $\Theta$. On moze biti svakojak kao sto vidimo u primjerima: postotak pripadnika jedne subpopulacije u populaciji, broj, varijanca neke statistike i mnoge druge oblike.

Na temelju uzorka iz populacije donosimo zakljucke. Buduci da sami uzorci podlijezu sansi (npr. nas uzorak FERovaca sastoji se od 20 zena no prezentira li to sliku populacije?) zelimo znati koliko su sigurne nase pretpostavke.