\chapter{Statističko testiranje na prosjek uzorka}

U ovom poglavlju cu prikazati kako se testiranje vrsi na prosjek uzoraka. Prvo pretpostavljamo da je populacija normalno distribuirana ili barem priblizno normalno distribuirana.

Zatim zelimo odrediti koliki nam mora biti uzorak iz te populacije da bi znacajnost testa te njegova snaga bila zadovoljavajuca i uz to sve izbalansirati. 

\marginnote{ok, kako odrediti velicinu uzorka? Ako znamo varijancu mozemo pretpostaviti da ce biti distribuiran s $\mathcal{N}(\mu + \delta, \sigma^2)$ te onda odrediti preciznost $\delta$, $\beta$ i onda iz toga $n$.}

Treba naravno napomenti da ce uzorak biti distribuiran po T distribuciji jer nam varijanca populacije nije poznata te pretpostavljamo normalnu distribuciju populacije.

Ukoliko je $H_0$ tocna nasa statisika: \[\frac{\sqrt{n} (\hat{\mu} - \mu)}{s} \sim T_{n-1}\]

slijedi $T$ distribuciju s $n-1$ stupnja slobode. Nakon malo matematike uz pretpostavku $H_0$:

\[
\hat{\mu} \sim \frac{s}{\sqrt{n}} \cdot T_{n-1} + \mu
\]

Uz znacajnost testa $\alpha$ dobivamo gornji i doljni interval pouzdanosti kao:

\[
\hat{\mu} \in \left< \frac{s}{\sqrt{n}} \cdot T_{n-1, \frac{\alpha}{2}} + \mu, \frac{s}{\sqrt{n}} \cdot T_{n-1, 1 - \frac{\alpha}{2}} + \mu \right>
\]

gdje $T_{n, \alpha}$ gdje pretstavlja $\alpha$ percentila $T$ distribucije s $n$ stupnjeva slobode.

Potom uz odredeni $\alpha$ napravimo intervale pouzdanosti za ovaj test. Ukoliko je prosjek uzorka unutar tog intervala nas test nije odbio $H_0$ te je presuda null hipoteza, dok u suprotnom biramo $H_a$ tj. altrenativnu hipotezu.


\marginnote{Treba ovdje uzeti neki primjer podataka i lijepo ih obraditi}[5cm]

Kako bi bili u mogucnosti efokasno i bezbolno raditi ovaj tip testova sljedeci programski kod pisan u pythonu te koristeci numpy biblioteku radi za nas:

\begin{figure}[H]
\begin{minted}{python}
from pylab import *
from numpy import *
from scipy.stats import t, norm
from numpy.random import normal

#Ovo je primjer ulaznih podata za testiranje
data = normal(3.1, 2, 15)
mu_test = 3
alpha = 0.05


n = data.size        # Broj uzoraka
s = data.std(ddof=1) # Procjena standardne devijacije populaije (ne uzoraka)
mu_s = data.mean()   # Procjena prosjeka iz uzoraka
mu_s_s = s/sqrt(n)   # Standardna devijacija prosjeka uzoraka

#Plottanje cisto da vidimo ulazne podatke
hist(data, bins=7);

(l,u) = t.interval(1-alpha, n - 1, mu_test,  mu_s_s)
print (l,u) #Interval pouzdanosti

if l < mu_s and  mu_s < u:
    print ("H_0")
else:
    print ("H_a")
\end{minted}
\end{figure}