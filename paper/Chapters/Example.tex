\chapter{Statističko testiranje očekivanja}

U ovom ću poglavlju prikazati kako se testiranje vrši na očekivanje. Pretpostavka je testa normalna distribucije populacije.

Prvi je korak određivanje veličine uzorka iz te populacije da bi značajnost testa te njegova snaga bila zadovoljavajuća, no konkretni postupci određivanja veličine uzorka izlaze izvan okvira ovog seminara. 

Kod testova na očekivanje s povećanjem uzorka nam je potrebna manji nivo značajnosti, $\alpha$ za istu preciznost. Naravno i sam $\beta$, tj. vjerojatnost greške 2. vrste se smanjuje i u graničnom postupku teže ka idealiziranom slučaju $\alpha = \beta = 0$.

Treba naravno napomenuti da će sam uzorak biti distribuiran po $t$ distribuciji jer nam varijanca populacije nije poznata te ju aproksimiramo iz samog uzorka.  

Ukoliko je $H_0$ točna statistika: \[\frac{\sqrt{n} (\bar{x} - \mu)}{s} \sim t_{n-1}\]

slijedi $t$ distribuciju s $n-1$ stupnja slobode. Nakon malo matematike uz pretpostavku $H_0$ \cite{engstat} \cite{vis3}:

\[
\bar{x} \sim \frac{s}{\sqrt{n}} \cdot t_{n-1} + \mu
\]

Uz značajnost testa $\alpha$ dobivamo područje prihvaćanja $H_0$ kao \cite{matstat}:

\[
\bar{x} \in \left<\mu - \frac{s}{\sqrt{n}} \cdot t_{n-1, \frac{\alpha}{2}}, \mu + \frac{s}{\sqrt{n}} \cdot t_{n-1, \frac{\alpha}{2}} \right>
\]

gdje $t_{n, \alpha}$ gdje predstavlja $1-\alpha$ percentila $t$ distribucije s $n$ stupnjeva slobode, prikazano u prethodnom poglavlju.

Potom uz određeni $\alpha$ napravimo područje prihvaćanja za ovaj test. Ukoliko je prosjek uzorka unutar tog intervala naš test nije odbio $H_0$ te je presuda nul hipoteza, dok u suprotnom presuđujemo $H_a$.

U nastavku je dan primjer programskog koda koju korištenjem pythona te biblioteka NumPy i SciPy odrađuje opisani postupak testiranja.

\begin{figure}[H]
\begin{minted}{python}
from pylab import *
from numpy import *
from scipy.stats import t, norm
from numpy.random import normal

#Ovo je primjer ulaznih podata za testiranje
data = normal(3.1, 2, 15)

mu_test = 3 #Ocekivanje iz nul hipoteze
alpha = 0.05 #Nivo znacajnosti ovog testa

n = data.size        # Broj uzoraka
s = data.std(ddof=1) # Procjena standardne devijacije populaije
mu_s = data.mean()   # Procjena prosjeka iz uzoraka
mu_s_s = s/sqrt(n)   # Standardna devijacija prosjeka uzoraka

(l,u) = t.interval(1-alpha, n - 1, mu_test,  mu_s_s)
print (l,u) #Podrucje prihvacanja

#Testiram je li prosjek uzoraka unutar podrucja prihvacanja ovog testa
if l < mu_s and  mu_s < u:
    print ("H_0")
else:
    print ("H_a")
\end{minted}
\end{figure}