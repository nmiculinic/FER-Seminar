\chapter{Struktura testa}

\section{Greske}

Nakon uvoda u osnovne pojmove vezane uz hipotezu pogledajmo s kolikom sigurnoscu mozemo tvrditi istinost nase prosudbe u tesu. Imamo 4 moguca ishoda:

% http://www.tablesgenerator.com/

\begin{table}[h]
\resizebox{\textwidth}{!}{%
\begin{tabular}{cc|c|c|}
\cline{3-4}
 &  & \multicolumn{2}{c|}{Nul hipoteza je} \\ \cline{3-4} 
 &  & Tocna & Netocna \\ \hline
\multicolumn{1}{|c|}{\multirow{2}{*}{Presuda testa je:}} & Odbaci & \begin{tabular}[c]{@{}c@{}}Greska 1. vrste\\ Lazno pozitivni\\ $P = \alpha$\end{tabular} & Tocno \\ \cline{2-4} 
\multicolumn{1}{|c|}{} & Prihvati & Tocno & \begin{tabular}[c]{@{}c@{}}Greska 2. vrste\\ Lazno negativni \\ $P = \beta$\end{tabular} \\ \hline
\end{tabular}
}
\end{table}

Kakav god test odabrali da potvrdimo ili odbacimo $H_0$ on moze rezultirati u greskama 1. ili 2. vrste. Evo jednog ilustrativnog primjera za gresku 1. vrste.

\subsection{Greske 1. vrste}

Odbacivanje nul hipoteze kada je ona tocna naziva se greskom 1. vrste. \cite{engstat}. Radi daljnjeg pojasnjenja pogledati primjer u nastavku:

Situacija je sljedeca. Vi ste cuvar nekoga sela i vasa je duznost oglasiti uzbunu ukoliko se vuk priblizava. Time je $H_0$ vuka nema. E ukoliko vuka stvarno nema, a vi ste oglasili uzbunu vi ste nacinili pogresku prve vrste ilitiga lazna pozitivnost.

Slicnu scenarij mozete vidjeti i s testom za trudnocu. Ukoliko krecete od hipoteze  $H_0:$ \emph{Nema trudnoce} te $H_a:$ \emph{trudnoca} te stvarno niste trudni, ali test pokazuje trudnocu to je jos jedan primjer Type I greske. Ona se oznacava s grckim slovom $\alpha$ te se naziva nivo znacajnost testa \textit{(eng. significance level)}.

Pri samoj konstrukcji statistickog testa ukoliko je $H_0$ tocna mozemo lijepo ustimati $\alpha$ na prihvatljivu granicu te ga mozemo lijepo ustimati jer cesto pretpostavljamo kako funkcija razdiobe izgleda. \marginnote{Ovdje dodati jos par slika i lijepse pojasniti}

\subsection{Greske 2. vrste}

No dobro, ovo je sve super, mozemo podesiti test da nam je $\alpha \approx 0$ no sto time dobivamo? Tu u pricu ulaze greske 2. vrste koje imaju vjerojatnost $\beta$. Ne odbacivanjem nul hipoteze, $H_0$, kada je ona netocna se naziva pogreska 2. vrste. \cite{engstat}

Koristeci primjere iz prethodne sekcije, vas test presudi da vuka nema dok on stvarno dolazi pred vasa vrata. Takoder vi ste trudni dok test za trudnocu to ne pokazuje dok nije prekasno. Faktor $\beta$ je povezan s pojmom \emph{snaga testa} \textit{(eng. power)} koja iznosti $1-\beta$. Ona je definirana kao vjerojatnost da ce test odbiti $H_0$ kada je ona lazna. 

No taj faktor $\beta$ je cesto nemoguce odrediti bez nekih pretpostavki. Npr. ((ovdje ubaciti neku normalnu distribuciju i koliko je beta ako je $\mu  + \delta$ te kako on ovisi. Malo matematike i formulu upisati))

%Malo lijepse se izraziti oko "balansacije"
Kod izrade svakog statistickog testa dolazimo do biranja omjera snage i znacajnosti istoga. Oni se ponasaju kao na klackalici, sto je veca snaga testa to je veca znacajnost i obratno. Idealni test bi imao snagu 1 te znacajnost 0, no to u praksi nije moguce, te time treba pazljivo odabrati ta dva parametra tjekom izrade samoga testa.

\section{P vrijednosti}
Sljedeci vazan pojam u ovom kontekstu je p vrijednost. Ona se definira kao najmanja znacajnost test, $\alpha$, za koju bi ovaj uzorak presudili odbacivanjem nul hipoteze, $H_0$ \cite{engstat}

\section{Podrucje prihvacanja}

Potom zelimo konstruirati podrucje prihvacanja za nas test. \cite{vis3} Ponovimo imamo $H_0$ hipotezu koja pretpostavlja neke parametre o distribuciju koju zelimo testirati. 

Neka je slucajna varijabla $X \sim \mathcal{D}$. 
Prvo definirajmo oznaku kao u vecini literatura \cite{vis3} \cite{engstat} 

$x_\alpha$ je $1-\alpha$ percentil te zadane distribucije.

Za zadani nivo znacajnosti $\alpha$ to zapravo znaci  $P(x \in \left<l, u\right>) = 1-\alpha$. Tj. da sansa da ukoliko je $H_0$ tocna vjerojatnost tocne presude iznosti $1-\alpha$

Najcesce se rabe tri tipa podrucja prihvacanja:
\begin{itemize}
	\item Gornji: $x \in \left< x_\alpha, +\infty \right>$
	\item Doljni: $x \in \left<-\infty, x_{1-\alpha} \right>$
	\item Dvostrani: $x \in \left<x_{1-\frac{\alpha}{2}}, x_{\frac{\alpha}{2}} \right>$
\end{itemize}