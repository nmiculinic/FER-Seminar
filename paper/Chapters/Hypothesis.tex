\chapter{Hipoteza}

Statisticka hipoteza je tvrdnja o parametru jedne ili vise populacija.\cite{matstat} Npr. tvrdnja da prosjecan broj jabuka pojedenih na dan iznosi 2 je validna hipoteza.

\begin{itemize}
	\item Nul hipoteza \textit{(eng. null hypothesis)} $H_0$
	\item Altrenativna hipoteza \textit{(eng. altrenative hypothesis)} $H_a$
\end{itemize}

\section{Nul hipoteza}
Nul hipoteza je nasa osnova pretpostavka o populaciji. Ona kao takva podlijeze statistici uzorka te populaciju. Zasto uzorka? Evo jedan primjer. Zelimo provjeriti hipotezu iz uvoda \emph{na FERu je 99\% muskaraca}. Nerealno je i skupo ici od svakog pojedinca na fakultetu, provjeriti s kojeg je on faksa te zapravo analizirati svakog pripadnika populacije. 

Time se bavi deskriptivna statstika, dok statisticka testiranja ulaze u inferencijsku statistiku. 

Zato radimo uzorak od $n$ primjeraka iz populacije te na temelju njih zakljcujemo za cijelu populaciju.

\section{Altrenativna hipoteza}
%Testiranje ne radi je li nesto tocno ili nije. Ja vidim samo crno bijelo kroz naocale dok ono u stvarnosti moze biti i rozo. U postupku nessto prihvacamo ili odbacujemo s nekim odredenim rizicima. Bez obzira na ishod testa.

No sto ako postupkom testiranja odbacujemo nul hipotezu? Onda prihvacamo altrentivnu hipotezu. To naravno ne implicira u apsolutnu tocnost jer na temelju rezultata dobivenih u uzorku ne mozemo nikad biti sasvim sigurni je li ponudena
hipoteza ispravna ili ne. \cite{vis3}

Neka je $H_0:$ \emph{Varianca bodova na SISu je 15}. tj. $H_0:\; \sigma^2 = 15$. Sto bi bila altrenativna hipoteza? 

Logicno negacija $H_0$ te je time $H_a:\; \sigma^2 \ne 15$. Ovo je dvostrana altrenativna hipoteza jer parametar $\sigma^2$ u altrenative moze biti i veci i manji od nul hipoteze. 

Postoji jos jedana mogucnost altrenativne hipoteze, a to je jednostrana. Glasi ovako: $H_a:\; \sigma^2 > 15$ ili $H_a:\; \sigma^2 < 15$. Kao sto vidimo ova altrenativna hipoteza gleda samo jednu stranu toga parametra te se zato zove jednostavna. \cite{engstat}

Kada rabimo koju? Jednostranu rabimo samo ako smo iznimno uvjereni da je suprotan slucaj nemoguc, zbog fizikalnih, matematickih ili ostalih zakona. Stoga najcesce uzimamo dvostranu altrenativnu hipotezu.